\chapter{Methoden zur Abschätzung der maximalen Versorgungslänge}


%-------------------------------------

\section{Lotus Domino}
\label{sec:2methoden}

%In diesem Kapitel wird auf die Modelle und Methoden eingegangen, die bei einem\\Modellierungsvorgang durchlaufen werden. 

%============================
\subsection{Gesamtmodell-Input Netzwerkgraph}
\label{sec:2 modelle}
%
%\vspace{0.5cm}
%Die Ausführungen zum Thema Gesamtmodell, welche in diesem Kapitel behandelt werden, beziehen sich auf \cite{pinkafeld1}, sofern nicht anders angegeben.
%\\ 
%\par Ein Anschlussbereich wird wie folgt als Netzwerkgraph dargestellt. $G=(V,E)$ mit $V$ als Menge von Knoten und $E$ als Menge der Kanten. 
%Die Erstellung von Knoten und Kanten wird unter der Berücksichtigung der Geodaten-Attribute aus Kapitel 2.2.1 realisiert.
%
%\vspace{0.3cm}
%Jeder Knoten $v\in V$ besitzt die folgenden Attribute: 
%
%
%
%\begin{itemize}
%\item geographische Position $x-Koordinate, y-Koordinate ~in~ der~ digitalen~Katastralmappe ~$
%\item Informationsbedarf $\pi(v) \in \mathds{N}_{(0)},v \in V~$
%\end{itemize}
%
%
%Der Informationsbedarf wird auf Basis der Mikrozellendaten identifiziert. Besitzt ein Knoten keinen Informationsbedarf, dann wird er mit $~\pi(v)=0~$belegt.
%Man spricht von einem spatialen Knoten, welcher nur aufgrund der Graphen-Konstruierung in der digitalen Katastralmappe (DKM) enstanden ist.\\
%Besitzt ein Knoten einen Informationsgehalt, gilt die Forderung $\pi(v)>0$, es handelt sich um ein Anschlussobjekt.
%Es gilt zu beachten, ob das Anschlussobjekt mit Kupferkabeln oder aber mit Glasfaserkabeln versorgt wird. Wird das Anschlussobjekt mit Kupferkabeln in
%das Netzwerk integriert, dann gibt der Informationsbedarf die Anzahl der Teilnehmeranschlusseinrichtungen wie zum Beispiel von Telefonen an. 
%Diese Teilnehmer werden mit kurzen Stichleitungen, sogenannten \textit{drop-wires} versorgt. 
%Wird der Teilnehmer mit Lichtwellenleitern angebunden, so gibt der Informationsbedarf die Anzahl der Wellenlängen wieder, für welche eine entsprechende
%Anzahl von \textit{drop-wires} gedemultiplexed werden müssen. 
%Kanten $e \in E$ können auch Verbindungstrassen von spatialen Knoten und Anschlussobjekten sein. 
%

%---------------------------------------------
\subsubsection{Workflow}
\label{sec:2 modelle}
%
%
%\vspace{0.5cm}
%
%\par Nachfolgend wird der Workflow eines Modelldurchlaufes beschrieben. Dieser wird grundsätzlich in zwei Schritten ausgeführt.
%Im ersten Schritt wird ein Top-down Clustering durchgeführt, auf welches eine Bottom-up Optimierung folgt.
%Um einen Modelldurchlauf zu ermöglichen, müssen die Inputdaten erstellt werden. Dies wird im nachfolgenden Kapitel 2.1.2 erläutert. 
%%---------------------------------------------
%\vspace{1cm}

\subsection{Normierte Geodaten}
\label{sec:2 modelle}
%
%
%\vspace{0.5cm}
%
%{\em Normierte Geodaten stellen die generalisierten, topologisch korrekten Nutzungsflächen der\\ digitalen Katastralmappe (DKM) dar, die um für die
%Graphgenerierung relevante Flächen wie Straßenkreuzungsflächen und Querungsflächen (Eisenbahnübergänge, Brücken über Gewässer) automatisiert
%bzw. semiautomatisiert erweitert werden}\cite{prossegger1}.
%
%\ Aufgrund von räumlichen Analysen und auf Vertriebsinformationen basierend, werden branchenspezifische Umsatzpotentiale für Standorte der
%Anschlussobjekte bestimmt. Hierbei werden die Faktoren der bestehenden Netzinfrastruktur und die nutzbare Infrastruktur in den 
%Geobasisdatensatz integriert. Als bestehende Netzwerkinfrastruktur versteht sich das Strom- und Glasfasernetz, und unter nutzbarer Infrastruktur
%werden unter anderem existierende Leerverrohrungen, welche es zu nutzen gilt, verstanden. Die Umsetzung erfolgt mittles eines FME (Feature Manipulation
%Engine)-Modells, in welchem eine automatisierte Aufbereitung des jeweiligen Gebietes auf Basis der definierten Geobasisdaten erfolgt\cite{prossegger1}.
%
%\begin{figure}[H]
%    \centerline{\includegraphics[scale=0.5]{pics/dkm.png}}
%    \caption[Digitale Katastralmappe]{\label{FiG:Digitale Katastralmappe }
%	Ausschnitt der digitalen Katastralmappe\cite{prossegger1}}
%\end{figure}



%---------------------------------------------
\subsubsection{Top-Down Clustering}
\label{sec:2methoden}
%
%
%\vspace{0.5cm}
%
%Um Netzwerke zu optimieren ist es sinnvoll diese in Hierarchien einzuteilen. Eine Unterteilung in ein hierarchisches Netzwerk wird in Folge beschrieben. 
%Dafür wird der Netzwerkgraph $G$ in Teilgraphen $H_{i}\subset G,
%~i=1,...,m$ zerteilt. Jeder dieser entstandenen Teilgraphen spiegelt nun einen Accessnetzbereich (eine Ebene tiefer) wieder. Weiters wird jeder entstandene
%Teilgraph $H_{i}$ weiter zerteilt in $K_{i,j}\subset H_{i}, ~j=1,...,n_{i}$ und es ergibt sich nun das Accesszellennetz. Man erhält $K_{i,j}$ für die
%$j$-te Accesszelle im $i$-ten Accessnetz vom Teilgraphen $H_{i}$. 
%\par Somit wurde ein Netzwerkgraph in eine Drei-Ebenen-Hierarchie unterteilt.
%Dieser Vorgang wird als Top-Down Clustering des Anschlussbereiches $G$ bezeichnet.
%\vspace{0.5cm}
%
%In diesem Abschnitt der Modellierung wird die Implementierung eingebettet. Es wird ein zusätzlicher Parameter eingeführt, welcher die Begrenzung der 
%maximalen Anschlusslänge reguliert. So kann der Input des Routing-Modells kontrolliert beziehungsweise reguliert werden.
%

%---------------------------------------------


\subsubsection{Bottom-up Optimierung}
\label{sec:2methoden}
%
%
%\vspace{0.3cm}
%
%Um eine Simulation bezüglich der Kosten optimieren zu können, wird das Netzwerk vom Grund auf neu, also Bottom-up simuliert.
%Die Teilgraphen $T'$ stellen kostenoptimierte, mit dafür entsprechenden Modellen berechnete, Netzwerktrassierungen dar.
%
%%-----------------------------------------------

\vspace{1cm}
\subsection{Cluster-Modell}
\label{sec:2 modelle}
%
%
%\vspace{0.5cm}
%
%\par Wählamtsbereiche, die als Ausgangspunkt dieser Berechnung dienen, werden durch Zerlegung des Ausgangsgraphen in Beziehung zur den vorhin definierten
%Ebenen (Kapitel 1.3) unterteilt. 
%Mit Hilfe einer Toolbox, welche im bestehenden Programm implementiert ist, werden Wählamtsbereiche in kleinere Subgebiete, den Accessnetzen, unterteilt. 
%Die sich ergebenden Accessnetzebenen werden abermals unterteilt, wobei man als Output die Accesszellebenen erhält. Der gesamte Vorgang der Zerlegung eines
%Gebietes in kleinere Subgebiete wird als Clustern bezeichnet und verläuft in einer Top Down Strategie.
%
%\par Hierbei wird für einen Netzwerkgraphen $G=(V,E)$ eine Zerlegung in Teilgraphen\\ vorgenommen. Teilgraphen werden als $G' =(V,E)$ bezeichnet.
%Durch die Dilatation eingeschränkt, wird diese Graphenzerlegung durchgeführt. 
%\vspace{0.3cm}
%
%\par Die Dilatation wird formal wie folgt beschrieben: 
%\\
%\begin{equation}
%\sum_{i=1}^{m}  max \left\lbrace\| u-v \|_{2} ~:u,v \in V_{i} \times V_{i} ~ mit ~\pi(u) >0,\pi(v) >0\right\rbrace
%\end{equation} 
%\\
%Es gilt die Teilgraphanzahl $m$und den minimalen Teilgraphdurchmesser zu finden,  wobei die maximale Versorgungslänge auf den Teilgraphdurchmesser beschränkt ist.
%\\
%\begin{equation} 
%~max~\left\lbrace\| u-v \|_{2}~:u,v \in V_{i} \times V_{i} ~ und~\pi(u)>0,\pi(v)>0\right\rbrace\leq ~ d_{max} ~f\ddot{u}r~ i=1,...,m  
%\end{equation} 
%\\
%Weiters gilt es, die Bedingung des Gesamtinformationsbedarfs zu berücksichtigen. Die Summen der Teilinformationsbedürfnisse aller Teilgraphen dürfen den Gesamtinformationsbedarf nicht überschreiten.
%\\
%\begin{equation}
%\sum_{v\in V_{i}} \pi(v) \leq \pi_{max}  ~f\ddot{u}r~  i=1,...,m ~   
%\end{equation} 
%\\
%\begin{figure}[H]
%    \centerline{\includegraphics[scale=0.5]{pics/Cluster_Abbildung.png}}
%    \caption[Bereichsunterteilung in Cluster]{\label{FiG:Bereichsunterteilung in Cluster } 
%	Bereichsunterteilung nach dem Cluster-Modell, ersichtlich ist die Einteilung in Accessnetze (grün) und Accesszellen (blau)\cite{tech_rep_1} }
%\end{figure}
%
%Es gibt zwei unterschiedliche Möglichkeiten das Clustern durchzuführen:
%
%
%\begin{itemize}
%\item Barrier Clustering
%\item Dens Clustering
%\end{itemize}
%
%
%\par Wird die Forderung (2.1 Dilatation) dadurch ersetzt, dass alle Kanten, welche die Teilgraphen $H_{i}$
%miteinander verbinden, zu einer oder auch mehreren Nutzungsklassen gehören müssen, so wird vom Barrier-Clustering Model gesprochen.
%Für dieses Modell ist die Lösung heuristisch. Beim Verwenden von Barrier-Clustering werden die Teilgraphen durch
%Breitensuche solange initialisiert, bis man auf Kanten mit vorgegebenen Nutzungen stößt.
%
%\vspace{0.5cm}
%\par  Verwendet man Dens-Clustering, kann für ein gegebenes $m$ eine exakte Lösung gefunden werden.  
%Wird die Lösung von den Forderungen 2.2 oder 2.3 verletzt,  dann wird $m$ solange erhöht, bis sie wieder erfüllt sind.
%Im Anschluss werden die Teilgraphen vereinigt, bis eine der Bedingungen, 2.2 oder 2.3, verletzt wird.
%
%
%\vspace{1cm}

% --------------------------------------------------------------------------

\subsection{Routing-Modell}
\label{sec:2 modelle}

%
%\vspace{0.5cm}
%
%\par Nach dem Clustern der unterschiedlichen Ebenen in Citynetz, Accessnetz  und Accesszellen ist es nun möglich Trassen in den einzelnen Teilbereichen zu
%berechnen. Beeinflussend auf die Wahl der Trassierung wirken Faktoren wie zum Beispiel Kosten für Trassenführung. Diese Kosten setzen sich aus mehreren
%Faktoren, wie zum Beispiel unter anderem den Widmungsklassen, zusammen. 
%\\
%Ergo ist es günstiger eine Verlegetrasse durch ein unbebautes Gebiet zu erstellen, während im
%innerstädtischen Bereich die Kosten dafür wesentlich höher sind.
%Diese sind in den Bereichen ländlich und innerstädtisch signifikant unterschiedlich und wirken nun auf die Erstellung eines Trassenverlaufes ein.\\ Besondere
%Rücksicht wird unter anderem auf Gewässer genommen. 
%Die Kosten, um diese zu durchqueren, sind im Vergleich zu einer Trassenführung auf festem Boden,
%wesentlich höher.
%Um dem entgegenzusteuern wird dieser Parameter per default hoch angesetzt. 
%Aufgrund dessen kann es vorkommen, dass es einzelne Anschlussobjekte gibt, welche über einen überdurchschnittlich langen Weg angebunden werden müssen
%\cite{tech_rep_1}.
%
% \vspace{0.5cm}
% 
%\par Um solche Anschlusslängen zu vermeiden, werden die Versorgungslängen bereits im Cluster-Modell berücksichtigt. Somit können sich im Routing-Modell
%solch lange Trassenführungen nicht mehr ergeben, da das Längenmaximum bereits zuvor festgelegt wurde. Die Anschlussobjekte müssen nun einem anderen 
%Cluster zugeteilt werden, um eine Optimierung bezüglich der Längenbegrenzung zu gewährleisten.\\
%Nachdem die Routen für alle Teilbereiche (Bottom Up) kostenoptimiert, bezüglich der Trassierungslänge, erstellt wurden, sind die
%Routinginformationen in vorgefertigten Dateien aufbereitet.\\
%Das Routing-Modell berechnet die Trassierung auf Basis des gegebenen Graphen. Dieser Graph wird vom Cluster-Modell aufbereitet und bereitgestellt. 
%Es ist nun möglich eine kostenoptimierte Trassierung in den einzelnen Subgebieten eines Wählamtsbereichs zu simulieren. 
%Dabei wird eine Bottom Up – Strategie verfolgt. Das bedeutet, dass zuerst alle Accesszellen, anschließend alle Access- und danach Citynetze 
%geroutet werden.
%In jeder dieser Hierarchien werden die Kanten so ausgewählt, dass in jedem Anschlussbereich
%
%\begin{enumerate}
%		\item[a.] genau eine Verbindung zwischen jedem enthaltenen Anschlussobjekt und einem Versorgungszentrum oder 
%		\item[b.] zwei knotendisjunkte Verbindungen für jedes Paar von Anschlussobjekten zum Versorgungszentrum existiert
%\end{enumerate}
%
%und die Summe der Kosten der gewählten Kanten ein Minimum ist.		
%
%
%
% 
%\vspace{1cm}
%% --------------------------------------------------------------------------

\section{Mathematisches Konzept}
\label{sec:2methoden}
%
%
%\vspace{0.5cm}
%
%Für nachfolgendes Kapitel ist es notwendig einige Begriffe vorab zu erklären.

\subsection{Baum}
\label{sec:2methoden}
%
%\vspace{0.5cm}
%
%Ungerichtete Graphen können in Form eines Baumes dargestellt werden.
%\vspace{0.3cm}
%
%Ein ungerichteter Graph $G$ ist dann ein Baum, wenn:
%
%\begin{enumerate}
%\item $G$ ist zusammenhängend und $m=n-1$;
%\item$G$ enthält keinen geschlossenen Weg und $m=n-1$;
%\item In $G$ gibt es zwischen jedem Paar genau einen Weg;
%\end{enumerate}
%
%\vspace{0.3cm}
%
%\begin{graybox}
%\textbf{\textit{Definition:}} Ein ungerichteter Graph $G=(V, E, \gamma)$ heißt Wald, wenn $G$ kreisfrei ist, also keinen elementaren Kreis besitzt. Falls$~G~$ zusätzlich zusammenhängend ist, so heißt $G$ Baum\cite{krumke1}.
%\end{graybox}
%
%\vspace{1cm}
%
%\begin{flushleft}
%Desweiteren besteht ein Baum aus folgenden Elementen:
%\end{flushleft}
%
%
%\begin{itemize}
%\item Kanten
%\item Knoten
%\end{itemize}
%
%Knoten sind über Kanten miteinander verbunden. Wurzel und Blätter sind Sonderformen von Knoten. Die Wurzel ist der Startpunkt eines Graphen. 
%Blätter hingegen sind die Enden eines Baumes, zu ihnen führt nur eine Kante. 
%
%\vspace{1cm}
%\subsection{Steiner Baum}
%\label{sec:2methoden}
%
%
%\vspace{0.5cm}
%
%Ist $G=(V, E )$ ein vollständiger Graph mit Kantengewichten $c: E \to \mathbb{R}_{+}$, so scheint bei flüchtiger Betrachtung das Problem, einen
%gewichtsminimalen Steiner Baum für die Terminalmenge $K$ zu finden, identisch damit zu sein, einen minimalen spannenden Baum im (vollständigen) 
%induzierten Subgraphen $G[K]$ zu bestimmen. Eine nähere Betrachtung zeigt aber, dass ein minimaler spannender Baum in $G[K]$ nicht zwangsweise ein minimaler Steiner Baum ist.
%\\
%
%
%\vspace{0.5cm}
%
%\begin{graybox}
%\textbf{\textit{Definition:}} Sei $G=(V, E )$ ein ungerichteter Graph und $K \subseteq V$ eine beliebige Teilmenge der Eckenmenge. Ein Steinerbaum
% in $G$ für die Menge$K$ ist ein Teilgraph $T \sqsubseteq G$, der ein Baum ist und dessen Eckenmenge $K$ umfasst: $K \subseteq V(T)$.
%Die Elemente von $~K~$ nennt man Terminale, die Ecken aus $V(T) \setminus K~ Steinerpunkte$\cite{krumke1}.
%\end{graybox}
%
%\vspace{1cm}


\subsection{Vorabschätzung über MST Ansatz}
\label{sec:2methoden}

%
%\vspace{0.5cm}
%
%Derzeit wird ein Steiner Baum zum Bestimmen der Trassierung verwendet. Mittels MST soll nun eine Abschätzung noch vor dem Routing-Modell 
%durchgeführt werden. \\
%Unter einem MST versteht sich ein minimal spannender Baum eines Graphen. \\ 
%
%\begin{graybox}
%\textbf{\textit{Definition:}} Sei $G=(V, E, \gamma)$ ein ungerichteter Graph. Ein Partialgraph $H=(V,E', \gamma)$ von $G$ ist ein spannender Baum, 
%wenn $H$ ein Baum ist. \\
%Jeder spannende Baum von $G$ enhält mindestens $|V(G)| -1$ Kanten\cite{krumke1}.
%\end{graybox}
%
%\vspace{0.5cm}
%
%Jeder Graph besitzt auch ein Gewicht. Das Gewicht setzt sich aus der Summe der
%Kantengewichte zusammen.  Alle Knoten des Hauptgraphen müssen auch im MST enthalten sein (Abbildung 2.3). 
%Ein Baum heißt minimal spannend, wenn kein anderer Spannbaum im selben Graphen, mit einem geringeren Gewicht existiert\cite{krumke1}.
%
%\begin{figure}[H]
%    \centerline{\includegraphics[scale=0.5]{pics/minimal_spannender_baum}}
%    \caption[Minimal Spannender Baum-MST]{\label{FiG:Minimal Spannender Baum }
%	Hauptgraph\cite{krumke1} und der Minimal Spannende Baum zu diesem \cite{krumke1}}
%\end{figure}	
%
%\par Mittels des Prim-Algorithmus wird ein minimal spannender Baum erstellt. 
%Als Output\\ergibt sich ein Untergraph  $G'(V',E')$, mit den günstigsten Kanten und Knoten, welche benötigt werden, um den minimal spannenden Baum für den 
%ursprünglichen Graphen $G(V,E)$ bestimmen zu können\cite{krumke1}.
%
%\vspace{1cm}

\subsection{Leaf Backward Algorithmus}
\label{sec2:methoden}
%
%
%\vspace{0.5cm}
%
%Der Output aus Kapitel 2.2.2 stellt nun einen minimal spannenden Baum (MST) dar.\\ Spannende Bäume existieren nur in zusammenhängenden ungerichteten
%Graphen.\\
%In diesem MST sind noch Knoten enthalten, welche keinen Informationsbedarf
%beinhalten und auch nicht als Verbindungsknoten dienen.
%
%\vspace{0.3cm}
%
%Nun müssen diese Bedingungen zutreffen um solche Knoten zu entfernen:
%
%\begin{enumerate}
%	\item ist der Informationsgehalt $\pi(v) = 0 $	
%	\item hat der Knoten nur einen Nachbarknoten, handelt es sich um ein Blatt
%\end{enumerate}
%
% 
%Es gilt für $v\in V$ folgende Bedingungen zu berücksichtigen:
%
%
%\begin{enumerate}
%  \item \textit{$v$ hat mehr als einen Nachbarn} 
%  \item \textit{$v$ ist $\pi(v)>0$ , also ein Anschlussobjekt} \label{l2}
%  \item \textit{$v$ ist als Distribution Center gekennzeichnet} \label{l3}
%\end{enumerate}
%
%Verletzt ein Knoten im Graphen eine oder mehrere dieser Bedingungen, so wird er und die zuführende Kante aus dem Graphen entfernt. 
%
%
%Der Output-Graph entspricht nun einem Steiner Baum.
%
%
%%==================
\subsection{Implementierungskonzept}
\label{sec:2methoden}

%
%\vspace{0.5cm}
%
%Die Implementierung setzt sich aus den folgenden Teilschritten zusammen. 
%
%\begin{itemize}
%  
%  \item Berechnen der günstigsten Wege im Graphen - MST Algorithmus
%  \item Reduzieren der redundanten Knoten mittels Leaf Backward Algorithmus 
%   \item Berechnen des Flächenschwerpunktes und bestimmen des Distribution Centers
%    \item Vergleich von Output Steiner Baum und MST
%\end{itemize}
%
%\begin{figure}[H]
%    \centerline{\includegraphics[scale=0.5]{pics/diagramm_1}}
%    \caption[Ablaufdiagramm für Einbindung in bestehendes Programm]{\label{FiG:Ablaufdiagramm für Einbindung in bestehendes Programm }
%	Ablaufdiagramm für Einbindung in bestehendes Programm}
%\end{figure}
%
%\vspace{0.5cm}
%
%%============
\subsection{Implementierung}
\label{sec:2methoden}
%
%\vspace{0.5cm}
%
%Da die Einbindung in das Programm RTR\_R2008a nicht Teil dieser Arbeit ist, wird der Vorgang abgeändert durchgeführt. 
%Es wird ein Testbed erstellt, in welchem die Resultate des MST-Ansatzes mit den Ergebnissen des Steiner Baumes verglichen werden.
%Wie im Ablaufdiagramm (Abbildung 2.5) ersichtlich, werden nun die Schritte in Pseudo-Codes beschrieben. Da das Cluster-Modell und die Aufbereitung der
%Geodaten bereits implementiert sind, werden diese im ersten Schritt nicht weiter beachtet. Somit wird mit der Berechnung des MST begonnen. Als Input 
%dient hierfür ein $.ist$ File, in welchem sich die Geodaten befinden. 
%
%\begin{figure}[H]
%    \centerline{\includegraphics[scale=0.5]{pics/diagrammRealisierung}}
%    \caption[Ablaufdiagramm der Realisierung]{\label{FiG:Ablaufdiagramm der Realisierung} Ablaufdiagramm der Realisierung}
%\end{figure}
%
%       
%\begin{figure}[H]
%    \centerline{\includegraphics[scale=0.7]{pics/diagrammRealisierungDetail}}
%    \caption[Ablaufdiagramm-Detail der Realisierung]{\label{FiG:Ablaufdiagramm-Detail der Realisierung} Detailansicht der realisierten Implementierung}
%\end{figure}
%       
%       
\subsubsection{MST Algorithmus}
\label{sec:2methoden}
%
%\vspace{0.5cm}
%
%Um einen minimal spannenden Baum zu erhalten gibt es mehrere Möglichkeiten.  In dieser Arbeit wird der Algorithmus von Prim verwendet, 
%um einen MST zu erhalten.
%In diesem Algorithmus bilden die ausgewählten Kanten zu jedem Zeitpunkt einen Baum  $G=(V,E)$. Zu Beginn wird ein 
%beliebiger Knoten $v \in V$ gewählt. Durch weiteres Hinzufügen der günstigsten Kanten, entsteht der gesuchte Baum\cite{turau1}. 
%\vspace{0.3cm}
%\par Die Auswahl der Kanten wird dabei wie folgt getroffen:
%
%\begin{enumerate}
% \item Sei $V$ die Menge der Ecken des gesuchten Baumes $G$ und $E$ die Menge der Ecken des Graphen.
% \item Man wähle unter den Kanten $(u,v)$, deren Anfangsecke $u \in V$ und deren Endecke $v \in E \setminus V$ liegen, diejenige mit der kleinsten
% Bewertung aus. 
%\item Diese wird dann in $G$ eingefügt.
%\item Weiters wird $v$ in $V$ eingefügt.
%\item Dieser Schritt wird solange wiederholt bis $V=E$ ist.
% \end{enumerate}
%
%
%Beim Algorithmus von Prim wird das Sortieren der Kanten vermieden, jedoch ist der Aufwand der Kanten aufwändiger. Die Auswahl und Verwaltung
%der Menge $V$ kann mit Hilfe eines Feldes erfolgen. Im Feld wird für jede Ecke $e \in E\setminus V$ die kleinste Bewertung unter den Kanten $(e,v)$ mit 
%$v \in V$ gespeichert. Es wird auch die Endecke dieser Kante abgespeichert.
%Wurde eine Kante $(e,v)$ ausgewählt wird die Ecke $e$ als zugehörig 
%von $V$ markiert.
%Die Laufzeit der Prim-Prozedur beträgt $n-1$ Iterationen, da bei jedem Durchlauf genau eine Ecke eingefügt wird. Der Aufwand für die 
%Auswahl der Kante und das Aktualisieren des Feldes beträgt $O(n)$. Das bedeutet, der Gesamtaufwand beträgt $O(n^2)$.
%Der Primalgorithmus sollte Verwendung finden, wenn die Anzahl der Kanten $m$ in etwa der Größenordnung von $n^2$ entspricht\cite{turau1}. 
%
%
%\vspace{0.5cm}
%
%\begin{lstlisting}[label=Prim-Algorithmus, caption=: Prim-Algorithmus, mathescape] 
%	var
%		U: set of Integer;
%		u,e: Integer;
%	begin
%		B.initBaum(G);
%		U:={1};
%		while $~U.anzahl \not= n~$ do begin
%		$~Sei ~(e,u)~ die~ Kante~ aus~ G~ mit~ der~ kleinsten~ Bewertung,~$ 
%	    $~so ~dass~ u \in U~ und~ e \in E\setminus U~$ 
%	    $~B.einfuegen(e,u);~$ 
%	    $U.einfuegen(e);~$ 
% 				end 
% 		end
%\end{lstlisting}
%
%\vspace{0.5cm}
%
%Der Output ist, wie in Kapitel 2.2.2 beschrieben, der kostengünstigste Untergraph.
%Es folgt der Schritt des Leaf Backward Algorithmus, in welchem die redundanten Knoten des Graphen entfernt werden.
%
%\vspace{1cm}
%
%

\subsubsection{Leaf Backward Algorithmus}
\label{sec:2methoden}
%
%\vspace{0.5cm}
%
%In diesem Algorithmus wird ein schrittweises Terminieren der Knoten $v$ mit $\pi(v)=0$ und der inzidenten Kanten $e$, zu diesem Knoten, verfolgt.
%Neben  Entfernen der Knoten aus dem MST, muss auch der Grad des Vorgängerknotens vermindert werden. 
%
%\vspace{0.4cm}
%
%\begin{lstlisting}[label=Leaf-Backward-Algorithmus, caption=: Leaf-Backward-Algorithmus, mathescape]
%Für alle $~v\in V~$ führe aus:
%var
%		g:    Integer; 			// Grad des Knoten
%		v,e: Integer;	
%	
%while
%	if $(Nachbar(v)=1~ und~ \pi(v)=0 ~und ~istDistr(v)=false)$
%		g.Nachbar(v)-1;
%		delete $v$;
%		delete $e$;
%	end 
%end
%\end{lstlisting}
%
%\vspace{0.5cm}
%

\subsubsection{Gewichteter Flächenschwerpunkt}
\label{sec:2methoden}
%
%
%\vspace{0.5cm}
%
%In jedem Teilbereich ist ein Wählamt, oder auch Distribution Center genannt, notwendig.
%Um einen geeigneten Standpunkt für dieses festlegen zu können, muss der gewichtete Flächenschwerpunkt ermittelt werden.
%Ein ungerichteter Graph $G(V,E)$ bildet einen Teilbereich unserer Netzstruktur ab.  Hierbei kann sich um eine Accesszelle oder ein Accessnetz
%eines Wählamtsbereiches (Kapitel 1.2) handeln. Für die Ermittlung des Schwerpunktes muss ein Startknoten $u \in V $  gewählt werden\cite{krumke1}. 
%\par Für seine Wahl sollte die Position so gewählt werden, dass er bestmöglich einem Distribution Center entsprechen kann. Zu diesem Zeitpunkt
%der Modellierung befindet man sich im Top-Down Clustering (Kapitel 2.1.3), darum kann die Wahl des potentiellen Distribution Centers nur 
%abgeschätzt und noch nicht errechnet werden. Die exakte Bestimmung eines Distribution Centers kann erst im Routing-Modell (Kapitel 2.1.4) bestimmt werden.
%
%\vspace{0.5cm}
%
%Die Position eines Knoten $u\in V$ im Graph $G$ besitzen $x$- und $y$-Koordinaten, welche Teil aus $ \mathbb{R} $ sind. Diese werden in einer Matrix
%dargestellt.	
%\\
%\begin{equation}
%		y =
%			\begin{bmatrix}
% 				v_{x}^1, & v_{y}^1, & v_{x}^2, & v_{y}^2, & \cdots,& v_{x}^n, & v_{y}^n
%			\end{bmatrix}
%\end{equation}
%\\
%Mit der Anzahl $n$ der Anschlussobjekte ($\pi(v)>0$) $u\in V$. Darum werden alle $x-~$und$~y-~$ Koordinaten von allen Anschlussobjekten des
%Graphen in den Vektor $y$ gespeichert.
%
%\vspace{0.3cm}
%Es wird eine Matrix $C \in Mat(2n\times 2)$ definiert.
%	\begin{equation}
%		C =
%			\begin{bmatrix}
% 				1 & 0 \\
% 				0 & 1 \\
% 				\vdots & \vdots
% 			\end{bmatrix}
%	\end{equation}
%	
%	
%Es wird eine Gütefunktion $F(p)$ mit gegebenem Vektor $y$ und gegebener rechteckiger Matrix $C$ formuliert. Gesucht wird ein Vektor $p\in 
%\mathbb{R}^{2}$, welcher die Gütefunktion
%
%\begin{equation}\label{e:guetefunktion1}
%		F(p) = \frac{1}{2} \| y - Cp \|^2 = \frac{1}{2} \sum_{i=1}^{n} [(v_{x}^i 
%- p_{x})^2 + (v_{y}^i - p_{y})^2]
%\end{equation} \\
%minimiert. Der Vektor $y - Cp$ entspricht der zu minimierenden Abweichung. Zur Lösung der Problemstellung wird die \textit{Methode der kleinsten 
%Quadrate} angewandt \cite{papageorgiou}. Dieses Problem kann optimal gelöst werden.
%
%\begin{equation}
%	p^* = (C^T C)^{-1} C^T y
%\end{equation} 
%
%Dieser Vektor $p^* = \begin{bmatrix}p^*_{x} \\ p^*_{y} \end{bmatrix}$ enthält die x- und y- Koordinaten des geographischen Schwerpunktes des Graphen $G$, 
%bezüglich aller enthaltenen Anschlussobjekte. Somit kann die Position des Distribution Centers optimal bestimmt werden.
%
%
