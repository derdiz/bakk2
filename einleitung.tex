

\chapter{Einleitung}
%=================================================

\section{Allgemeines}
\label{sec:1einleitung}


\vspace{0.5cm}
\par Die Firma Uniquare ist ein Softwarehaus mit dem Hauptfirmensitz in Krumpendorf/Österreich. Dieses Unternehmen 
hat sich auf die Entwicklung von Banken-Software spezialisiert. \linebreak
Uniquare Software Development GmbH verwendet als Workgroup-Software Lotus \linebreak Notes/Domino von IBM\textsuperscript{\copyright}.
Mit dieser Software werden annähernd die gesamten \linebreak administrativen Prozesse, sowie auch Projektabwicklungen im Unternehmen verwaltet und
\linebreak unterstützt.\\
Lotus Notes/Domino wird fälschlicherweise oft als ein E-Mailsystem verstanden, jedoch handelt es sich hierbei um ein System, welches 
mehr bereitstellt. Dieses System stellt unter anderem Datenbanken, Workgroup-Software, Web-Server etc. zur Verfügung. Es
eignet sich um firmeninterne Kommunikationsprozesse zu vereinfachen und diese zu beschleunigen\cite{muhs/klatt}.


\vspace{0.3cm}
%=================================================


\section{Aufgabenstellung}
\label{sec:1einleitung}

\par Die sich im Unternehmen in Verwendung befindenden Datenbanken wurden seit der Einführung von Lotus Notes im Betrieb fortlaufend entwickelt.  
Lotus Notes und später auch Lotus Domino, werden seit der Version 2.0 in diesem Unternehmen verwendet. \newline
Da nicht alle Kunden Lotus Notes/Domino n\"utzen, gibt es den Wunsch, auf projektbezogene Datenbanken \"uber einen Browser
zugreifen zu k\"onnen. Somit kann sich der Kunde selbst, aktiv am Projektablauf beteiligen. 
Die technischen M\"oglichkeiten ohne Lotus Notes auf Domino Applikationen
zuzugreifen waren bis zum Release der Version 4.x nicht gegeben.\\
Das Ziel dieser Arbeit ist es, eine Schablone für die Überführung von Datenbanken (Version 4.x zu 8.5) zu erstellen.
Aus diesem Grund wird ein Template (eine Vorlage), unter der Verwendung der XPages-Technologie, erstellt.
Mit diesem Template kann in weiterer Folge, unter geringem Aufwand, eine Überführung auf die Version Lotus Notes/Domino 8.5.2 durchgeführt werden.



\vspace{0.8cm}

%\footnote{Unter Workgroup oder Groupware verstehen sich die Bemühungen in Forschung
%und Praxis, arbeitsteilige Prozesse mit Hilfe von Informations-und Kommunikationstechnologien zu unterstützen und effektiver zu
%gestalten\cite{metzler}.}
%\footnote{Zum Beispiel können Anforderungen zu den zuständigen Abteilungen, 
%in einem selbst definierten, automatisierten Ablauf weitergeleitet werden.}
%Firmeninterne- sowie Kunden-Befragungen ergaben jedoch, dass der Zugriff auf Datenbanken in ihren Funktionen nicht benutzerfreundlich
%und nicht mehr zeitgemäß ist. 
%Ab der Version 8.x von Lotus Notes/Domino und der Einführung der XPages-Technologie, wird das Erstellen einer Anwendung für Web und Notes Client
%nicht mehr unterschieden.\\
%Firmeninterne Befragungen ergaben, dass eine Vereinfachung der Bedienung, wie in etwa die eines Internet-Browsers, gewünscht sei.\\
 %Diese Schablone nutzt die mit XPages überzuführen.